\chapter{INTRODUCTION}
\label{chpt:introduction}
%%%Follow this structure: https://www.scribbr.com/category/dissertation/
\section{Background}
A broad range of industries use sensors, detectors, and transducers of many kinds to test, measure, and control various business processes. Specifically, recent developments in the Internet of Things (IoT) make sensors a primary tool to provide enhanced automation and production. Therefore, there is a massive amount of multivariate proximal sensor data available, and there is a pressing need for technologies to analyze and utilize this enormous amount of data. For instance, the soil is an essential element of life. Understanding soil's physical and chemical properties is of high interest for soil scientists, farmers, policymakers such as the United States Department of Agriculture (USDA) and Natural Resources Conservation Service (NRCS), or other government agencies such as the National Aeronautics and Space Administration (NASA). Collecting data and characterizing soil properties in large geographical areas using conventional laboratory procedures are often high-cost, time-consuming, and environmentally unfriendly (due to involving destructive chemicals). Also, accurate soil health assessments require many different types of measurements. Thus, researchers have struggled to establish an effective unified method for quantifying soil health~\cite{Wang2015Synthesized, pham2019soaviz}. 

Recently, using proximal sensors such as portable X-ray fluorescence spectroscopy (\pxrf{}), visible - near-infrared (\visnir{}) spectroscopy, color sensor (e.g., Nix Pro\footnote{https://www.nixsensor.com/}) to analyze soil horizons is gaining favor with the ability to provide faster, non-destructive data acquisition from soils. They offer a rapid means of quantifying of physical and chemical properties of soil samples. In other words, recent advancements in the fields of proximal sensor devices and related technologies offer rapid, cost-effective, and environmental-friendly alternatives to soil profile characterizations~\cite{Legoria2012, Paulette2015, Stockmann2016, Guilherme2018, sun2020enhanced}. In other words, with careful sampling and analytical protocols, a large proximal sensor dataset (e.g., acquired using pXRF or other proximal sensor devices) can be produced rapidly. Also, based on the principle of matching analytical methodology with the nature of the geochemical problem under investigation~\cite{paul2010fit}, such datasets can provide valuable insights~\cite{gazley20113d}. Established literature shows that these insights regarding the soil using proximal sensor data contribute significantly to agricultural productivity and environmental assessment activities.

%%%%%%%%%%%%%%%%%%%%%%%%%%%%%%%%%%%%%%%%%%%%%%%%%%%%%%
\section{Motivations for an intelligent visual analytics approach}

While the data collection time using proximal sensor devices reduces significantly, the data analysis tasks are still a burden and often take the soil scientists a large portion of time~\cite{pham2019soaviz}. For instance, the typical approach to analyzing pXRF soil pedon scanning results is using Microsoft Excel~\cite{Zhu2011}. Though using Microsoft Excel is convenient, it is challenging to study more complex data. Also, some advanced software packages such as Global Mapper (Blue Marble Geographics, Hallowell, ME), ArcGIS (ESRI, The Redlands, CA), NCSS 8 (NCSS, Kaysville, UT)~\cite{Paulette2015,Guilherme2018}, MDI Jade v9.1.1~\cite{Chakraborty2017}, GeoChem, and SAGA GIS~\cite{Curi2018} are also popular. Still, these advanced packages require a reasonable training time before being able to use them. In many cases, soil scientists even need to use complicated programming languages/packages like MatLab, R, and Python to analyze their data~\cite{Wesseling2013}. In other words, existing software approaches are ill-suited to explore this type of multivariate proximal sensor data. Specifically, existing analysis solutions are simple graphics, provide unintuitive visualizations, lack support for interactions, or require skills or training before use.

Additionally, the analysis tasks are time-consuming because they require inter-disciplinary expertise. Precisely, besides domain expertise, soil scientists often need data analysis skills to explore and find insights from the multivariate proximal sensor data acquired in their business domain. As a result, soil scientists often use simple techniques (e.g., using Excel) to analyze their data and use basic, static visualizations to report the results to the public~\cite{pham2020soilscanner}. Simple analysis solutions are often unintuitive due to the lack of data visualization components. Similarly, the basic, static visualizations do not provide interactions to support exploration while searching insights from the underlying multivariate data.

In analyzing data, visualization is invaluable for communicating complex data (large, multivariate, or time series). Visualization is becoming more critical when researchers seek insights or when they need to share their findings with end-users with a diversity of expertise such as policymakers and the general public~\cite{theriot2020tailoring}. Furthermore, besides static graphics, interactions provide a significant advantage in offering domain experts the ability to investigate their data further using their expertise intuitively. In other words, human understanding, exploration, and analysis of multivariate proximal sensor datasets can be enhanced by interactive visualizations~\cite{balla2017possibilities}. Human cognitive ability is limited, and without appropriate visualizations, it is hard for humans to consume a large amount of data.

Soil analysts cannot often generate meaningful visualizations for their acquired data. Likewise, data analysts often do not understand the analysis tasks or the data generated in different business domains. Thus, the interdisciplinary, collaborative approach, in this scenario, proves to be successful. Specifically, data visualizers can produce appropriate visualizations generated based on analysis tasks specified by soil scientists. Furthermore, these visualization tools should also provide interactive options that allow the soil scientists to explore their data using their specialized domain expertise. In other words, though having the ability to provide appropriate data visualization solutions, data visualizers cannot do the data analysis by themselves but offer visualizations and interactive options for the scientists to perform the analysis tasks instead.

Characterizing soil profiles may be divided into two main categories. The first category focuses on a specific site (e.g., a pedon at a specific location). Contrariwise, the second one collects georeferenced data and analyzes them in a large geographical area. The former allows scientists to focus on analyzing extensive characteristics of soil pedons at a particular geographical location. In contrast, the latter enables the investigation of soil properties over large geographical areas incorporated with spatial information. This latter approach offers farmers and policymakers better insights into the land under analysis. A better understanding of the land is achieved via a broader overview of the large area of concern as a whole instead of analyzing individual, specific locations separately. Furthermore, whenever a particular site is of interest to the analysts, a cut plane revealing data in that particular location is also available for exploration in the latter approach.

The visual representations, customized for individual proximal sensor data analysis cases, vary depending on the data collection settings and analysis tasks at hand. Thus, it is nearly impossible to have standard visualizations to accommodate all the different users' needs while analyzing multivariate proximal sensor data. In this case, visual recommendations have a great potential to tackle these issues. Personal recommendations have gained success in several fields, such as movie recommendations (e.g., Netflix), commercial product recommendations (e.g., Amazon), and advertisement recommendations (e.g., Google Ads). However, personalized visualization recommendations are still limited. That said, this dissertation explores different approaches to visual recommendations that can learn and recommend appropriate visualizations to the users using extracted visual features and end-users contexts.

%%%%%%%%%%%%%%%%%%%%%%%%%%%%%%%%%%%%%%%%%%%%%%%%%%%%%%
\section{Motivations for machine learning methods in this area}

In analyzing the soil using proximal sensors, users often need to purchase different calibration packages or develop their calibration packages (e.g., using Excel, R, or Python) for few experts/capable users. Most current manufacturers use their conventional, low-level spectrum analysis to calibrate and quantify the detected data for different material types. In other words, there are difficulties with device calibrations, and there is a need for developing advanced methods (e.g., using machine learning and deep learning) for device calibration purposes. In the same vein, discussions with proximal sensor users imply that they desire to have machine learning models to predict higher-level properties (e.g., soil fertility, soil texture, lithographic properties) from elemental concentration data collected using the proximal sensors. These machine learning models can replace the need of going through slow, costly, and destructive (due to environmentally unfriendly chemicals) laboratory procedures to quantify these properties.

Discussions with soil scientists, NRCS representatives, and NASA representatives revealed that there are two main trends in using \pxrf{} or \visnir{} data for soil profile (or pedon interchangeably in Pedology terminology) analysis. The first is to analyze the data using analytical methods and visualizations. The second is to utilize the relatively easy-to-collect data using these devices and predict high-level, more difficult-to-measure soil properties as a fast, cost-saving, and environmentally friendly alternative to laboratory procedures. Existing work in this area often involves a small number of samples (40 to 400) localized in some particular geographic regions. Even with a small number of samples, the data in this area often features many data parameters (e.g., 216 to 1024 wavebands from \visnir{} spectrum in the range of 350 to 2,500 nm). For these reasons, conventional machine learning (ML) approaches to these problems have gained more popularity than their deep learning (DL) counterparts. Moreover, all of the existing ML/DL approaches have relatively low accuracy for several reasons, such as a small number of training samples, a large number of data features, model complexity, or simplicity. Therefore this dissertation explores different ML/DL techniques to predict soil properties from spectral data acquired from soil samples.

%%%%%%%%%%%%%%%%%%%%%%%%%%%%%%%%%%%%%%%%%%%%%%%%%%%%%%
\section{Objectives and contributions}

This dissertation first details the objectives for a software solution for analyzing multivariate proximal sensor data. These five objectives are the results of a National Science Foundation (NSF) I-Corps insights after interviewing 102 stakeholders in the fields that use multivariate proximal sensor data. These five objectives are:
(1) research and development of a set of common interactive visualizations for analyzing multivariate proximal sensor data; (2) research and development of an intelligent visual recommendation component that provides appropriate, personalized visualizations for the end-users based on the characteristics of the underlying multivariate proximal sensor data and the users' analysis tasks at hand; (3) research and development of real-life indications for potential errors exist in the underlying multivariate proximal sensor data; (4) research and development of machine learning components for proximal sensor device calibrations; and (5) research and development of machine learning components for soil/water property predictions using multivariate proximal sensor data acquired from them. These five objectives set the blueprints for the research discussed in this dissertation. Specifically, the contributions of this dissertation include:
\begin{itemize}
    \item A set of objectives for integrated visual analytics and machine learning approaches for analyzing multivariate proximal sensor data

    \item A set of interactive, 2D, and 3D visualizations for analyzing multivariate proximal sensor data acquired from soil profiles using \pxrf{} proximal sensor
    
    \item A set of interactive, 2D, and 3D visualizations for analyzing georeferenced multivariate proximal sensor data acquired from soil profiles over large geographical areas
    
    \item Initial research results for a static approach to visual recommendation using visual features extracted from the underlying multivariate proximal sensor data and a dynamic approach for this purpose using reinforcement learning
    
    \item Experiments with different machine learning and deep learning approaches for predicting soil properties from spectral data acquired using proximal sensors
\end{itemize}


The research foundations for this dissertation were completed with initial prototypes, conference/journal papers~\cite{pham2019soaviz, sun2020enhanced, pham2020soilscanner, pham2020scagnosticsjs, outliagnostics, pham2019mtsad}, and a patent (Title: DATA VISUALIZATION DEVICE AND METHOD; Number: WO/2021/055243; Priority Date: September 16, 2019)~\cite{pham2020patent}, and further submitted/accepted publications (one accepted book chapter, one accepted conference paper, one submitted journal paper, and one submitted conference paper). Moreover, validation of the importance of this invention was accomplished through the NSF I-Corps\footnote{https://www.nsf.gov/news/special\_reports/i\-corps/} (Award Number: 2017018), Spring 2020, Cohort \#3. The next phase is the research and development phase that leads to a proof-of-concept of this technology. This technological translation step will advance the existing prototype and further validate its importance with customers, propelling this innovation to the next development stage.

Finally, after the introduction (this chapter), this dissertation is organized as follows. Chapter \ref{chpt:userrequirements} describes the analysis requirements elicited from proximal sensor end-users and other stakeholders. After consolidating the requirements, Chapter \ref{chpt:visualizationsforpedon} details how data visualizers and soil scientists worked together and discovered a set of common interactive data visualizations suitable for analyzing multivariate proximal sensor data. Similarly, Chapter \ref{chpt:soilcoresvisualizations} details a set of interactive, 2D, and 3D visualizations for analyzing georeferenced multivariate proximal sensor data collected over large geographical areas. After Chapters \ref{chpt:visualizationsforpedon} and \ref{chpt:soilcoresvisualizations}, it is clear that it is not possible to create a complete set of interactive visualizations to analyze multivariate proximal sensor data for different domains or end-users. Therefore, Chapter~\ref{chpt:visualrecommendations} explores static and dynamic approaches for making visual recommendations as attempts to tackle this issue. Chapter \ref{chpt:abnormalitydetections} tackles two issues in this area; it works with time series form of multivariate proximal sensor data; it also tackles an important task that is detecting abnormalities in the underlying multivariate proximal sensor data. Furthermore, Chapter \ref{chpt:soilpropertypredictions} discusses different experimented ML and DL models to predict soil properties using spectral data. This chapter also describes a novel DL model, called RDNet, that achieves state-of-the-art results in predicting \phho{} and \phkcl{} from large-scale, globally distributed soil \visnir{} spectra. Finally, Chapters \ref{chpt:discussions} and \ref{chpt:conclusion} discuss the results and conclude this work respectively.